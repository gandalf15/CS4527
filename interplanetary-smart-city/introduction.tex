\chapter{Introduction\label{chap:introduction}}
\quad The challenges of cities are changing. As we use more of that finite resource of clean drinking water, as we create more waste and use more energy, we must think very differently how to solve the problems. In 2014 United Nations estimated that 54 percent of the world's population lived in urban areas and predicted that the number increases to 66 percent by 2050 \cite{noauthor_worlds_2014}. Air pollution in cities is proliferating. World Health Organization estimated that in 2012 died 3.7 million people because of outdoor air pollution exposure \cite{noauthor_who_2014}. What is a smart city? Is it a solution to these problems? In 2018 there is no agreed definition of a smart city. I would say that it is an urban area that produces and uses data from many sensors called Internet-of-Things (IoT) devices. These data collections help to improve life, health, comfort and resource management of the city. The answer is that it is not the solution to all problems that cities have. On the other hand, it can help to use resources more efficiently and bring new insights based on collected data to issues we are facing. One example can be an improvement in public transportation and traffic light control. This improvement can directly decrease air pollution. The same approach can be used in our homes, villages or even on a global and interplanetary scale. In 2024 first crew should begin their mission to Mars\footnote{\url{http://www.spacex.com/mars}} and set up first Mars base, from which we can build a city and eventually a self-sustaining civilisation. An interplanetary smart city will require communication technology that is secure, scalable, distributed and aware of the physical distance between information and request location.
\vspace{\baselineskip}

%We report on the development
This dissertation reports the development, testing and evaluation of Interplanetary Smart City (IPSC), a fully decentralised peer-to-peer (P2P) application designed to allow communication between IoT devices. This application can be deployed in a variety of scenarios from smart homes to interplanetary smart cities. The project aims to explore the possibilities of utilising blockchain technology, the best ideas from multiple (P2P) protocols and ideology of IPFS~\cite{labs_ipfs_nodate}. Such an application will bring numerous advantages in comparison to traditional cloud-based solutions\todo{ref}. IPSC will allow secure, robust, reliable and space aware communication without a central point of failure and possible savings on server hosting and cloud services.

\section{Motivation}
\quad Data are, supposedly, the currency of the Internet age. Companies are increasingly allowing payments for their digital services with information rather than money \cite{curtis_how_2015}. Existing technologies that are used in smart cities are mostly centralised and do not allow easy trade with collected data between other parties. This approach worsens all mentioned criteria for security, reliability, robustness and no physical distance awareness. On the other hand, it is beneficial to the companies providing these services because they have easy and usually free access to the data collections. In order to use such services as Oracle Cloud IoT\footnote{\url{https://cloud.oracle.com/iot}}, Google Cloud IoT\footnote{\url{https://cloud.google.com/solutions/iot/}}, Salesforce IoT\footnote{\url{https://www.salesforce.com/products/salesforce-iot/overview/}} or Microsoft Azure IoT\footnote{\url{https://www.microsoft.com/en-us/internet-of-things/azure-iot-suite}} our device must be connected to the Internet. This dependency can be a disadvantage because in a case of Internet connection failure we cannot access our data. Furthermore, if the service provider is hacked or experiencing technical issues, then our data collections can be stolen, inaccessible or permanently deleted. This poses high dependency on the service provider and Internet connectivity. One example of service provider dependency can be Logitech that decided to intentionally brick all Harmony Link devices remotely on 16 March 2018 \cite{noauthor_logitech_nodate}.
\vspace{\baselineskip}

The next issue is that IoT devices need to communicate each other often. The standard client-server model can introduce a bottleneck and increased latency. Moreover, it is a single point of failure. On the contrary, P2P network is ideal for a smart city. The workload is spread across multiple devices and requested data can be retrieved directly from the closest local node that is in possession of them. In the case of a future interplanetary city, the physical distance of required data is critical. The time required to travel radio wave a distance between Earth and Mars is approximately from 4.3 minutes up to 21 minutes. This depends on the position of the planets. Furthermore, interplanetary space is a very different environment. High-energy ionising particles (electrons, heavy ions and protons) of the space environ causes Single Event Effects (SSE) such as Single Event Upset, Single Event Transient, Multiple Bit Upset and many other destructive and nondestructive SSE. They are responsible for the arbitrary behaviour of electronics and corruption of memory \cite{duzellier_radiation_2005}. From these examples, it is clear that the current approach of IoT and a smart city communication is not suitable for the future use.
\vspace{\baselineskip}

\section{Objectives}
\quad The main research interest in this project is focused on the applications of P2P, Byzantine fault tolerant Blockchain Technology for an interplanetary smart city. The main project objectives are to develop, test and perform an initial evaluation of an application prototype that allows scalable, secure, robust, reliable and information distance aware communication for IoT devices. The project addresses a subtask of providing an easy way of data collection exchange between untrusted parties. (The goal of the IPSC research is, therefore, to enable...)
\vspace{\baselineskip}